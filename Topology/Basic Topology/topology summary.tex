\documentclass[a4paper]{book}
%\usepackage[spanish]{babel}
%\usepackage{fontenc}
%\usepackage[utf8]{inputenc}
\usepackage{hyperref}
\usepackage{fancyhdr}
\usepackage{graphicx}
\usepackage{amsmath}
\usepackage{amsfonts}
\usepackage{amssymb}
\usepackage{amsthm}
\usepackage{xcolor}
\usepackage{tcolorbox}
\tcbuselibrary{theorems}
\tcbuselibrary{breakable}
\tcbuselibrary{skins}
%\usepackage[left=30pt,right=40pt,top=40pt,bottom=60pt]{geometry}
\usepackage[bottom=60pt]{geometry}

\raggedbottom

\newtheoremstyle{ejemplo}% name
{5pt}% Space above
{5pt}% Space below
{}% Body font
{}% Indent amount
{\bfseries}% Theorem head font
{.}% Punctuation after theorem head
{.5em}% Space after theorem head
{}% Theorem head spec (can be left empty, meaning ‘normal’)

\newtheoremstyle{demostracion}% name
{5pt}% Space above
{5pt}% Space below
{}% Body font
{}% Indent amount
{\scshape}% Theorem head font
{.}% Punctuation after theorem head
{.5em}% Space after theorem head
{Proof}% Theorem head spec (can be left empty, meaning ‘normal’)

% TODO: Fix spacing 
\theoremstyle{definition}
\newtcbtheorem[number within = section]{define}{Definition}{enhanced jigsaw, breakable, before skip=10pt, after skip=10pt, sharp corners=all, colframe=purple, fonttitle=\bfseries%
	}{defn}	
	
\newtheorem{theorem}{Theorem}[section]
\tcolorboxenvironment{theorem}{%
	colframe=blue, enhanced jigsaw, breakable,%
	before skip=10pt,after skip=10pt, sharp corners=all}

\newtheorem{note}{Note}[section]

\theoremstyle{ejemplo}
\newtheorem{example}{Example}[section]
\newtheorem{remark}{Remark}[section]
\newtheorem{exercise}{Exercise}[chapter]

\theoremstyle{demostracion}
\newtheorem{dem}{Proof}
\tcolorboxenvironment{dem}{%
	colframe=black, opacityback = 0, enhanced jigsaw, breakable,%
	before skip=10pt,after skip=10pt, sharp corners=all}

\newcommand{\complejos}{\mathbb{C}}
\newcommand{\reales}{\mathbb{R}}
\newcommand{\racionales}{\mathbb{Q}}
\newcommand{\enteros}{\mathbb{Z}}
\newcommand{\naturales}{\mathbb{N}}
\newcommand{\realnmatrix}[1]{\mathcal{M}_{#1}(\reales)}
\newcommand{\quottop}{\stackrel{\sim}{\tau}}

\DeclareMathOperator{\apl}{Apl}
\DeclareMathOperator{\biy}{Biy}
\DeclareMathOperator{\simgroup}{\mathcal{S}}
\DeclareMathOperator{\interior}{Int}
\DeclareMathOperator{\exterior}{Ext}
\DeclareMathOperator{\closure}{Cl}
\DeclareMathOperator{\boundary}{Fr}
\DeclareMathOperator{\accumulation}{Ac}
\DeclareMathOperator{\sats}{sats}

%idr where did i get these
\makeatletter
%\def\thickhrulefill{\leavevmode \leaders \hrule height 1ex \hfill \kern \z@}
\def\@makechapterhead#1{%
  \vspace*{10\p@}%
  {\parindent \z@ \raggedleft \reset@font
            \scshape \@chapapp{} \thechapter
        \par\nobreak
        \interlinepenalty\@M
    \Huge \bfseries #1\par\nobreak
    %\vspace*{1\p@}%
    \hrulefill
    \par\nobreak
    \vskip 100\p@
  }}
\def\@makeschapterhead#1{%
  \vspace*{10\p@}%
  {\parindent \z@ \raggedleft \reset@font
            \scshape \vphantom{\@chapapp{} \thechapter}
        \par\nobreak
        \interlinepenalty\@M
    \Huge \bfseries #1\par\nobreak
    %\vspace*{1\p@}%
    \hrulefill
    \par\nobreak
    \vskip 100\p@
  }}

\pagestyle{empty}
\author{NyKi}
\title{Basic Topology}

\begin{document}
\maketitle
\tableofcontents

\pagestyle{fancy}
%%Thing
\chapter{Topological spaces}
TODO\\ 
Change all $\subseteq$ into $\subset$ (for clearer notation).\\ 
Make header better.
\section{Topologies and bases}

\begin{define}{Topological space}{top_space}
	A \textbf{topological space} is a pair $(X,\ \tau)$ where $X$ is a nonempty set and $\tau$ is a family of subsets of $X$, such that:
	\begin{enumerate}
		\item
		$X,\ \emptyset \in \tau$.
		\item
		Any arbitrary union of elements of $\tau$ is again in $\tau$.
		\item
		Any finite intersection of elements of $\tau$ is again in $\tau$.	
	\end{enumerate}	 
	The set $\tau$ is called a \textbf{topology} over $X$.
	Subsets of $X$ which are in $\tau$ are called \textbf{open} in $X$ and sets which their complement is in $\tau$ are called \textbf{closed} in $X$.
\end{define}

Note that a topology $\tau$ is never empty by axiom $1$ in our definition.
Generally, if the topology is clear we simply refer to the topological space as the set $X$, understanding that we are working with an implicit topology.

\begin{example}
	Over any nonempty set $X$ we can define the \textbf{discrete topology} by taking $\tau = \wp(X)$, where $\wp(X)$ is the set of all subsets of $X$ (the power set of $X$). It is easily checked that this satisfies every axiom of a topology.
\end{example}

\begin{example}
	$\tau = \{X,\ \emptyset \}$ is also a topology over $X$, called the \textbf{trivial topology}.
\end{example}

\begin{example}
	Given a point $p \in X$ in a nonempty set $X$ we can define the \textbf{special point topology} by taking $\tau = \{A \subseteq X\ : p \in A\} \cup \{ \emptyset \}$.
\end{example}

\begin{example}\label{top_euclidean}
	For a more interesting example, we can consider over the real numbers the topology $\tau = \{A \subseteq \reales\ :\ \forall x \in A\ \exists I \text{ open interval s.t. } x \in I \subseteq A\}$. This one is called the \textbf{euclidean topology} over the real numbers. As it is quite ubiquitous, whenever we talk about the real numbers without explicitly mentioning the topology we will assume they carry the euclidean topology.
\end{example}

\begin{define}{Comparison of topologies}{topology_comparison}
	Let $X$ be a nonempty set and let $\tau_1$ and $\tau_2$ be two topologies over $X$. If $\tau_1 \subseteq \tau_2$ we say that $\tau_2$ is \textbf{finer} than $\tau_1$ and that $\tau_1$ is \textbf{thicker} than $\tau_2$.
\end{define}

The first problem that we have is finding non trivial topologies. For that purpose, we introduce the concept of a base.

\begin{define}{Base}{base}
	Let $X$ be a nonempty set. $\mathcal{B} \subseteq \wp(X)$ is called a \textbf{base} (for a topology) over $X$ if
	\begin{enumerate}
		\item
		$\bigcup_{B \in \mathcal{B}} B = X$.
		
		\item
		For every $B_1,\ B_2 \in \mathcal{B}$ with nonempty intersection and $x \in B_1 \cap B_2$ there exists a $B_3 \in \mathcal{B}$ such that $x \in B_3 \subseteq B_1 \cap B_2$.
	\end{enumerate}
	An element of a base is called a \textbf{basic element}.
\end{define}
It is easily checked that condition 2 implies that when given a finite number of basic elements with nonempty intersection and $x$ in that intersection, there exists some other basic element that is between $x$ and the intersection, check exercise \eqref{exer_bases}.\\ 
Bases will allow us to characterize topological concepts using a family of sets much smaller than the family of open sets.
First, we see how we can get a topology from a base.

\begin{theorem}
	Let $X$ be a nonempty set and $\mathcal{B}$ a base over $X$. The set
	\begin{equation*}
		\tau_{\mathcal{B}} = \{O \subseteq X\ :\ \forall x \in O\ \exists B \in \mathcal{B} \text{ s.t. } x \in B \subseteq O\}
	\end{equation*}
	is a topology over $X$, called the \textbf{topology generated} by $\mathcal{B}$. Moreover, if $\tau'$ is another topology such that $\mathcal{B} \subseteq \tau'$ then $\tau_{\mathcal{B}} \subseteq \tau'$ (that is to say $\tau_{\mathcal{B}}$ is the smallest topology that contains $\mathcal{B}$).
\end{theorem}

\begin{proof}
	We first have to check that $\tau_{\mathcal{B}}$ satisfies the axioms for a topology.
	\begin{enumerate}
		\item
		$\emptyset \in \tau_{\mathcal{B}}$ trivially and $X \in \tau_{\mathcal{B}}$ because of $\bigcup_{B \in \mathcal{B}} B = X$
		
		\item
		Let $\{ U_i\}_{i \in I}$ be an arbitrary family of sets in $\tau_{\mathcal{B}}$, and let $U = \cup_{i \in I} U_i$. If $x \in U$ then $x$ is in at least one $U_i$, so there exists $B \in \tau_{\mathcal{B}}$ with $x \in B \subset U_i \subset U$ and so $U \in \tau_{\mathcal{B}}$.
		
		\item
		Let $\{ U_i\}_{i = 1}^{n}$ be a finite family of sets in $\tau_{\mathcal{B}}$, and let $U = \cap_{i = 1}^{n} U_i$. If $U$ is empty, then $U \in \tau_{\mathcal{B}}$. If $U$ is non-empty, and if $x \in U$ then $x$ is in every $U_i$ and because of exercise \eqref{exer_bases} there exists a $B \in \mathcal{B}$ such that $x \in B \subset U$ and so $U$ is in $\tau_{\mathcal{B}}$ by definition.
	\end{enumerate}
	So $\tau_{\mathcal{B}}$ is a topology. The second part of the theorem is a consequence of the next theorem.
\end{proof}

\begin{theorem}
	Let $X$ be a nonempty set and $\mathcal{B}$ a base over $X$. Then the topology generated by $\mathcal{B}$ is equal to the set of all possible unions of elements of $\mathcal{B}$ (taking the empty union to equal the empty set).
\end{theorem}

\begin{proof}
	
\end{proof}

\begin{example}
	The euclidean topology of example \eqref{top_euclidean} is generated by the base $\mathcal{B}$ consisting of all open intervals, that is sets of the form $]a,\ b[\ = \{ x\in \reales : a < x < b\}$.
\end{example}

Now that we know how to get a topology from a base, it's convenient to be able to get a base from a topology.

\begin{theorem}
TODO: See if it's correct
	Let $(X,\ \tau)$ be a topological space, and $\mathcal{B} \subseteq \wp(X)$ a set of subsets of $X$. Then $\mathcal{B}$ is a base for $\tau$ if and only if $\forall O \in \tau$ and $\forall x \in O$ there exists $B \in \mathcal{B}$ such that $x \in B \subseteq O$.
\end{theorem}

And finally, we have a result for comparing the topologies generated by two bases.

\begin{theorem}[Hausdorff Criterion]
	Let $X$ be a nonempty set and $\mathcal{B}_1,\ \mathcal{B}_2$ two bases over $X$. If for every $B_1 \in \mathcal{B}_1$ and every $x \in B_1$ there exists $B_2 \in \mathcal{B}_2$ such that $x \in B_2 \subseteq B_1$ then $\tau_{\mathcal{B}_1} \subseteq \tau_{\mathcal{B}_2}$.
\end{theorem}

We now introduce the concept of a subbase, a generalization of a base.

\begin{define}{Subbase}{subbase}
	Let $X$ be a nonempty set. A family of sets $\delta$ of $X$ is called a \textbf{subbase} for a topology over $X$ if the union of every set in $\delta$ equals $X$.
\end{define}

\begin{theorem}
	Let $X$ be a nonempty set, and $\delta$ a subbase over $X$. Then the arbitrary union of finite intersections of sets in $\delta$ form the smallest topology that contains $\delta$, such a topology is called the \textbf{topology generated} by $\delta$.
\end{theorem}

This theorem is the important thing for subbases: they are almost arbitrary families of sets of $X$, and they provide a way of talking about the smallest topology that contains them.

\begin{define}{Neighborhood}{neighborhood}
	Let $(X,\ \tau)$ be a topological space and $x \in X$. A subset $U \subseteq X$ is called a \textbf{neighborhood} if there exists $O \in \tau$ such that $x \in O \subseteq U$. If $U$ is then open, it's called an \textbf{open neighborhood}, and if $U$ is an element of some base, it's called a \textbf{basic neighborhood}. We denote the set of all neighborhoods of $x$ with $\mathcal{N}(x)$.
\end{define}

\begin{define}{Neighborhood base}{neighborhood base}
	Let $(X,\ \tau)$ be a topological space and $x \in X$. A set $\mathcal{B}$ of subsets of $X$ is called a \textbf{neighborhood base} if
	\begin{enumerate}
		\item
		Every element of $\mathcal{B}$ is a neighborhood of $x$.
		
		\item
		For every neighborhood $U$ of $x$ there exists $B \in \mathcal{B}$ such that $x \in B \subseteq U$.
	\end{enumerate}
\end{define}

\section{Closed and open sets}

Open and closed sets are important in the definition of a topology. It's no wonder that we study some of their properties.\\ 
First, we see that closed sets satisfy properties similar to that of open sets.

\begin{theorem}\label{closed_set_props}
	Let $(X,\ \tau)$ be a topological space. Then the set $\mathcal{F}$ of closed sets of $(X,\ \tau)$ satisfy:
	\begin{enumerate}
		\item
		$X,\ \emptyset \in \mathcal{F}$.
		\item
		Any finite union of elements of $\mathcal{F}$ is again in $\mathcal{F}$.
		\item
		Any arbitrary intersection of elements of $\mathcal{F}$ is again in $\mathcal{F}$.
	\end{enumerate}	
\end{theorem}

We can also define a topology via a family of sets that satisfy the conditions in theorem \eqref{closed_set_props}, see exercise .

\begin{define}{Closure}{closure}
	Let $(X,\ \tau)$ be a topological space and $A \subseteq X$. We define the \textbf{closure} of $A$ to be the intersection of all closed subsets of $X$ that contain $A$, and denote it by $\closure(A)$.
\end{define}

In the definition of closure we assume that the intersection of a bounded below by inclusion subset of closed sets exists, but this is always true due to item 3 in theorem \eqref{closed_set_props}. The closure of a nonempty set $A$ is nonempty as there will always be a closed set containing $A$ (the set $X$), so the family of closed subsets containing $A$ will be nonempty, and the intersection of all closed sets that contain $A$ will, inevitably, contain $A$.

\begin{define}{Limit and accumulation point}{limit_point}
	Let $(X,\ \tau)$ be a topological space. $p \in X$ is said to be a \textbf{limit point} of a nonempty set $A \subseteq X$ if all neighborhoods of $p$ intersect $A$. If all neighborhoods of $p$ intersect $A \setminus \{ p\}$ then $p$ is said to be an \textbf{accumulation point} of $A$.
\end{define}

\begin{define}{Isolated point}{isolated_point}
	Let $(X,\ \tau)$ be a topological space. $p \in X$ is said to be an \textbf{isolated point} of a nonempty set $A \subseteq X$ if $p \in A$ and there exists a neighborhood $U$ of $x$ such that $U \cap A = \{ p\}$.
\end{define}

\begin{define}{Interior and exterior}{interior}
	Let $(X,\ \tau)$ be a topological space and $A \subseteq X$. We define the \textbf{interior} of $A$ to be the union of all open subsets of $X$ contained in $A$, and denote it by $\interior{A}$. The \textbf{exterior} of $A$ is defined as $\exterior(A) = \interior(X \setminus A)$.
\end{define}

\begin{define}{Interior and exterior point}{interior_point}
	Let $(X,\ \tau)$ be a topological space. $p \in X$ is said to be an \textbf{interior point} of a nonempty set $A \subseteq X$ if $A$ is a neighborhood of $p$. If instead $X \setminus A$ is a neighborhood of $p$, $p$ is said to be an \textbf{exterior point} of $A$.
\end{define}

An interior point of a subset is also a limit point of the same subset. It may be believed that an interior point is also an accumulation point, but this is not always true:
\begin{example}
	Consider the set $X = \{ a,\ b,\ c\}$ with $a,b,c$ all distinct, and consider the discrete topology over $X$. Then, if $A = \{a,\ b \}$ it is easily checked that $a$ is an interior point of $A$ because $\{ a\}$ is open, but it is not an accumulation point because the open neighborhood $\{ a\}$ of $a$ doesn't intersect $A \setminus \{ a\} = \{ b\}$.
\end{example}

\begin{define}{Boundary}{boundary}
	Let $(X,\ \tau)$ be a topological space and $A \subseteq X$. We define the \textbf{boundary} (or frontier) of $A$ to be $\boundary(A) = \closure(A) \setminus \interior(A)$, and a point inside the boundary of $A$ is called a \textbf{boundary point}.
\end{define}

\begin{theorem}
	Let $(X,\ \tau)$ be a topological space and $A \subseteq X$. Then:
	\begin{enumerate}
		\item
		$\interior(A)$ is equal to the biggest open set that is contained in $A$.
		\item
		$\closure(A)$ is equal to the smallest closed set that contains $A$.
	\end{enumerate}
\end{theorem}

\begin{theorem}
	Let $(X,\ \tau)$ be a topological space and $A \subseteq X$.
	\begin{enumerate}
		\item
		$A$ is open if and only if $\interior(A) = A$. That is to say, $A$ is open if and only if it's a neighborhood of every point inside it.
		\item
		$A$ is closed if and only if $\closure(A) = A$.
	\end{enumerate}
\end{theorem}

\begin{theorem}
	Let $(X,\ \tau)$ be a topological space and $A \subseteq X$. Then:
	\begin{enumerate}
		\item
		$\interior(A)$ is equal to the set of all interior points of $A$.
		\item
		$\closure(A)$ is equal to the set of all limit points of $A$.
		\item
		$\exterior(A)$ is equal to the set of all exterior points of $A$.
	\end{enumerate}
\end{theorem}

\begin{theorem}
	Let $(X,\ \tau)$ be a topological space and $A,B \subseteq X$.
	\begin{enumerate}
		\item
		If $A \subseteq B$ then $\closure(A) \subseteq \closure(B)$ and $\interior{A} \subseteq \interior{B}$. Nothing can be said for the boundary.
		\item
		$\closure(A) \cup \closure(B) = \closure(A \cup B)$ and $\interior{A} \cup \interior{B} \subseteq \interior{A \cup B}$.
		\item
		$\closure(A) \cap \closure(B) \supseteq \closure(A \cap B)$ and $\interior{A} \cap \interior{B} = \interior{A \cap B}$.
		\item
		$\interior(A) = X \setminus \closure(X \setminus A)$.
	\end{enumerate}
\end{theorem}

\begin{theorem}
	Let $(X,\ \tau)$ be a topological space. If $A \subseteq X$, then
	\begin{equation*}
		X = \interior(A) \cup \boundary(A) \cup \exterior(A).		
	\end{equation*}
\end{theorem}

We can also characterize previous definitions via bases and neighborhood bases.

\begin{theorem}
	Let $(X,\ \tau)$ be a topological space, $\mathcal{B}$ a base for $\tau$ and $A \subseteq X$. If $p \in X$, then
	\begin{enumerate}
		\item
		$p$ is a limit point of $A$ if and only if every basic neighborhood of $p$ intersects $A$.
		
		\item
		$p$ is an interior point of $A$ if and only if there exists some basic neighborhood of $p$ contained in $A$.
		
		\item
		$p$ is a boundary point of $A$ if and only if every basic neighborhood of $p$ intersects $A$ and $X \setminus A$.
		
		\item
		$p$ is an accumulation point of $A$ if and only if every basic neighborhood of $p$ intersects $A$ at a point different from $p$.
		
		\item
		$p$ is an isolated point of $A$ if and only if $p \in A$ and there exists some basic neighborhood of $p$ that doesn't contain any other points of $A$.
	\end{enumerate}
\end{theorem}

\begin{theorem}
	Let $(X,\ \tau)$ be a topological space and $A \subseteq X$. If $p \in X$ and $\mathcal{B}(p)$ is a neighborhood base for $p$, then
	\begin{enumerate}
		\item
		$p$ is a limit point of $A$ if and only if every element of $\mathcal{B}(p)$ intersects $A$.
		
		\item
		$p$ is an interior point of $A$ if and only if there exists some element of $\mathcal{B}(p)$ contained in $A$.
		
		\item
		$p$ is a boundary point of $A$ if and only if every element of $\mathcal{B}(p)$ intersects $A$ and $X \setminus A$.
		
		\item
		$p$ is an accumulation point of $A$ if and only if every element of $\mathcal{B}(p)$ intersects $A$ at a point different from $p$.
		
		\item
		$p$ is an isolated point of $A$ if and only if $p \in A$ and there exists element of $\mathcal{B}(p)$ that doesn't contain any other points of $A$.
	\end{enumerate}
\end{theorem}


\section{Subspaces}
\begin{define}{Subspace topology}{subspace_topology}
	Let $(X,\ \tau)$ be a topological space, and $A \subseteq X$ nonempty. We define the \textbf{subspace topology}
	\begin{equation*}
		\tau_A = \{O \cap A : O \in \tau \}.
	\end{equation*}
	The resulting topological space $(A, \tau_A)$ is called a \textbf{subspace} of $X$.
\end{define}

\begin{theorem}
	The set defined in definition \ref{defn:subspace_topology} is a topology over $A$.
\end{theorem}

Note that we may equip a subset of a topological space with any topology we want, but that doesn't make it a subspace. The subspace topology is important as it is the one that ``makes'' sense to put on a subset, as if $A$ is an open set then $\tau_A \subseteq \tau$. We sometimes just refer to the subspace $(A,\ \tau_A)$ as $A$, just as with spaces.

\begin{theorem}
	Let $(X,\ \tau)$ be a topological space, and $A \subseteq X$ nonempty. The set of closed sets $\mathcal{F}_A$ of the space $(A,\ \tau_A)$ is
	\begin{equation*}
		\mathcal{F}_A = \{C \cap A : C \in \mathcal{F} \}.
	\end{equation*}
\end{theorem}

We have to be careful when using the vocabulary of the last two sections in relation to subspaces. It may happen that a set is open in $A$, but not be open in $X$, or be closed in $A$ and not be in $X$, or viceversa:
\begin{example}

\end{example}

We can however guarantee stronger transitivity under certain conditions:
\begin{theorem}
	Let $(X,\ \tau)$ be a topological space, and $A \subseteq X$ nonempty. Then
	\begin{enumerate}
		\item
		If $A$ is open in $X$, then $B \subseteq A$ is open in $A$ iff $B$ is open in $X$.
		
		\item
		If $A$ is closed in $X$, then $B \subseteq A$ is closed in $A$ iff $B$ is closed in $X$.
	\end{enumerate}
\end{theorem}
This is a crude explanation of why open and closed subspaces are more interesting than other subspaces, more reasons will be given in the topological invariants chapter. For now, we will give properties of subspaces related to the parent space.

\begin{theorem}
	Let $(X,\ \tau)$ be a topological space, $\mathcal{B}$ a base for $\tau$ and $A \subseteq X$ nonempty. Then
	\begin{equation*}
		\mathcal{B}_A = \{ B \cap A : B \in \mathcal{B}\}
	\end{equation*}
	is a base for the subspace topology over $A$.
\end{theorem}

\begin{theorem}
	Let $(X,\ \tau)$ be a topological space, $A \subseteq X$ nonempty and $x \in A$. Then
	\begin{equation*}
		\mathcal{N}_A(x) = \{ U \cap A : U \in \mathcal{N}(x)\}
	\end{equation*}
	is the family of neighborhoods of x in $(A, \tau_A)$.
\end{theorem}

\begin{theorem}
	Let $(X,\ \tau)$ be a topological space, $A \subseteq X$ nonempty, $x \in A$ and $\mathcal{B}(x)$ a neighborhood base for $x$ in $X$. Then
	\begin{equation*}
		\mathcal{B}_A(x) = \{ B \cap A : B \in \mathcal{B}(x)\}
	\end{equation*}
	is a neighborhood base of x in $(A, \tau_A)$.
\end{theorem}

We can also talk about the closure, interior, exterior and boundary of a set inside a subspace. For notational convenience, in the corresponding operators we use a subscript to indicate which space we are talking about when there could be confusion: for example, if $S \subseteq A \subseteq X$ the closure of $S$ inside $A$ would be denoted by $\closure_A(S)$, meanwhile if $Y$ is a topological space which is disjoint from $X$ the closure of a subset $H$ of $Y$ shall be denoted by $\closure(H)$, as there can be no confusion of which space we are in.

\begin{theorem}
	Let $(X,\ \tau)$ be a topological space, $A \subseteq X$ nonempty and $S \subseteq A$. Then
	\begin{equation*}
		\closure_A(S) = \closure(S) \cap A.
	\end{equation*}
\end{theorem}

This is quite nice compared to the cases of the interior and the boundary: TODO: Are they correct?
\begin{theorem}\label{interior_in_subspace}
	Let $(X,\ \tau)$ be a topological space, $A \subseteq X$ nonempty and $S \subseteq A$. Then
	\begin{equation*}
		 \interior(S) \cap A \subseteq \interior_A(S).
	\end{equation*}
	If $A$ is open, then the inclusion turns into equality.
\end{theorem}

\begin{theorem}
	Let $(X,\ \tau)$ be a topological space, $A \subseteq X$ nonempty and $S \subseteq A$. Then
	\begin{equation*}
		 \boundary_A(S) \subseteq \boundary(S) \cap A.
	\end{equation*}
\end{theorem}

The inclusion in theorem \eqref{interior_in_subspace} is not always an equality:
\begin{example}
	In the real numbers over the euclidean topology, consider the set of rational numbers $A = \racionales$. Then
	\begin{equation*}
		\interior(A) = \emptyset
	\end{equation*}
	but
	\begin{equation*}
		\interior_A(A) = A
	\end{equation*}
	so
	\begin{equation*}
		\interior(A) \cap A = \emptyset \subsetneq \interior_A(A) = \racionales.
	\end{equation*}
\end{example}

\section{Continuous functions}
\begin{define}{Continuous function}{cont_function}
	Let $(X,\ \tau)$ and $(Y,\ \tau')$ be topological spaces. An application $f: X \rightarrow Y$ is \textbf{continuous} with respect to the topologies $\tau$ and $\tau'$ if for every $O \in \tau'$ we have $f^{-1}(O) \in \tau$. When talking about applications between topological spaces, we denote them by $f: (X,\ \tau) \rightarrow (Y,\ \tau')$.
\end{define}


\section{Product spaces}
\section{Terminal topologies}

\begin{define}{Quotient space}{quotient_space}
	Let $(X,\ \tau)$ be a topological space and let $\sim$ be an equivalence binary relation (ERB). We define over the quotient set $X / \sim$ the \textbf{quotient topology}
	\begin{equation*}
		\stackrel{\sim}{\tau} = \{ O \subseteq X / \sim \ :\ p^{-1}(O) \in \tau\}
	\end{equation*}
	(which is indeed a topology). The space $(X / \sim,\ \stackrel{\sim}{\tau})$ is called the \textbf{quotient space} of $(X,\ \tau)$ under $\sim$. The application $p: (X,\ \tau) \rightarrow (X/\sim, \quottop)$ defined by $p(x) = [x]$ for all $x \in X$ is called the \textbf{canonical projection}.
\end{define}

\begin{theorem}
	The set $\stackrel{\sim}{\tau}$ defined in \eqref{defn:quotient_space} is a topology, and it is the finest topology for which the canonical projection $p: X \rightarrow X / \sim$ is continuous.
\end{theorem}

\begin{proof}
	It is obvious that $p:(X,\ \tau) \rightarrow (X/\sim,\ \quottop)$ is continuous. Now, let $\tau'$ be any other topology over $X/\sim$ in which $p$ is continuous. Then for every $O' \in \tau'$ we have $p^{-1}(O') \in \tau$, which means $O' \in \quottop$ and so $\tau' \subseteq \quottop$.
\end{proof}

\begin{theorem}
	A function $f: (X / \sim,\ \quottop) \rightarrow (Y,\ \tau')$ of a quotient space onto a topological space is continuous if and only if $f \circ p : (X,\ \tau) \rightarrow (Y,\ \tau')$ is continuous.
\end{theorem}

\begin{proof}
	If $f$ is continuous, then $f \circ p$ is a composition of continuous functions, so it is continuous. Now, suppose $f \circ p$ is continuous for some $f: (X / \sim,\ \quottop) \rightarrow (Y,\ \tau')$. If $O' \in \tau'$ then $(f \circ p)^{-1}(O') \in \tau$, but it is known that for preimages $\tau \ni (f \circ p)^{-1}(O') = (p^{-1} \circ f^{-1})(O') = p^{-1}(f^{-1}(O'))$ and by definition of the quotient topology $f^{-1}(O') \in \quottop$, so $f$ is continuous.
\end{proof}

\begin{theorem}
	The family of closed sets of $(X / \sim, \quottop)$ is exactly
	\begin{equation*}
		\stackrel{\sim}{\mathcal{F}} = \{ C \subseteq X/\sim\ :\ p^{-1}(C) \text{ is closed in } X\}.
	\end{equation*}
\end{theorem}

\begin{define}{Saturated set}{saturated_set}
	A set $S \subseteq X$ is called \textbf{saturated with respect to $\sim$} (or simply saturated) if 
	\begin{equation*}
		S = p^{-1} (p(S)).
	\end{equation*}
	We denote the set of all saturated subsets of $X$ by $\sats(X)$ (whenever the EBR is clear). Given $S \subset X$, the \textbf{saturation} of $S$ is defined as
	\begin{equation*}
		p^{-1} (p(S)).
	\end{equation*}
\end{define}

\begin{theorem}
	A set $S \subseteq X$ is saturated if and only if it's an union of equivalence classes.
\end{theorem}

\begin{theorem}
	\begin{equation*}
		\sats(X) = \{p^{-1}(p(S))\ : S \subseteq X \}
	\end{equation*}
\end{theorem}

\begin{theorem}
	Arbitrary unions and arbitrary intersections of saturated sets are saturated (with respect to the same EBR). Moreover, $X \in \sats(X)$ and $\emptyset \in \sats(X)$, so $\sats(X)$ is a topology over $X$.
\end{theorem}

\begin{theorem}
	Open sets in $X / \sim$ are exactly projections of open saturated sets, 
	\begin{equation*}
		\quottop = p(\sats(X) \cap \tau),
	\end{equation*}
	and closed sets in $X / \sim$ are exactly projections of closed saturated sets,
	\begin{equation*}
		\stackrel{\sim}{\mathcal{F}} = p(\sats(X) \cap \mathcal{F}).
	\end{equation*}
\end{theorem}

These results lead neatly to a characterization of when is the projection open or closed.
TODO: Characterization of open or closed for functions quotient to other space.
\begin{theorem}
Wrong?
	\begin{itemize}
		\item
		The canonical projection $X$ is open if and only if the saturation of every open set in $\tau$ is open.
		
		\item
		The canonical projection $X$ is closed if and only if the saturation of every closed set in $\tau$ is closed.
	\end{itemize}
\end{theorem}

\begin{theorem}
	For all $x \in X$
	\begin{equation*}
		\mathcal{N}([x]) = p(\{ U \in \sats(X)\ |\ \exists O_x \in \tau \cap \mathcal{N}(x) \cap \sats(X) \text{ with } x \in O_x \subseteq U\}),
	\end{equation*}
	which is a well defined set.
\end{theorem}

\begin{theorem}
	$(X/\sim,\ \quottop)$ is T1 if and only if every equivalence class is closed in $(X,\ \tau)$.
\end{theorem}

\begin{proof}
	By characterization of T1, 
\end{proof}

\begin{theorem}
	$(X/\sim,\ \quottop)$ is Hausdorff if and only if for every $x, y \in X$ with $x \not \sim y$ there exist saturated sets $O_x \in \tau \cap \mathcal{N}(x)$ and $O_y \in \tau \cap \mathcal{N}(y)$ such that $O_x \cap O_y = \emptyset$.
\end{theorem}

This condition is hardly operative.\\ 

One thing we may want is to find a space easier to work with that is homeomorphic to our quotient space.

\begin{theorem}
	Let $g: (X,\ \tau) \rightarrow (Y,\ \tau')$ be a continuous surjective function that induces an application $f$ on the quotient set $X/\sim$. If for all $O \subseteq Y$ we have that $g^{-1}(O) \in \tau$ implies $O \in \tau'$ then the application $f: (X/\sim,\ \quottop) \rightarrow (Y,\ \tau')$ is a homeomorphism.
\end{theorem}

Now, we can also understand an EBR over $X$ as a surjective function over some other set, $Y$ and viceversa:

\begin{theorem}
	Let $X$ and $Y$ be nonempty sets and $g: X \rightarrow Y$ a surjective application. Defining an EBR $\sim$ on $X$ by
	\begin{equation*}
		x \sim y \text{ iff } g(x) = g(y)
	\end{equation*}
	we get that $g$ induces a map on the quotient, $\stackrel{\sim}{g} : (X/\sim) \rightarrow Y$ which is a bijection. Also, for all $y \in Y$ we have $g^{-1}(y) = [x]$ for all $x \in g^{-1}(y)$.
\end{theorem}

If over some set $X$ we have an EBR $\sim$, this EBR uniquely defines the surjective application of projection over the quotient. This two results are purely set theoretical, and lead to an alternate definition of quotient topology, one more general using surjective functions:

\begin{theorem}
	Let $(X,\ \tau)$ be a topological space, $Y$ be a nonempty set and $g: X \rightarrow Y$ a surjective application. If we define a topology $\tau'$ over $Y$ by 
	\begin{equation*}
		\tau' = \{ O' \subseteq Y\ |\ g^{-1}(O') \in \tau\}
	\end{equation*}
	and an EBR $\sim$ on $X$ by
	\begin{equation*}
		x \sim y \text{ iff } g(x) = g(y)
	\end{equation*}
	we get that $g$ induces a map of spaces on the quotient, $\stackrel{\sim}{g} : (X/\sim,\ \quottop) \rightarrow (Y,\ \tau')$ which is a homeomorphism.
\end{theorem}

TODO: Kolmogorov, T1 and Hausdorff quotients

\section{Problems and exercises}
I have decided to compile interesting or useful exercises in this section. It is of note that they are not ordered in any way other than the order of the sections, and that I have not cared for the pedagogical value they provide so some may be too hard or others too easy. They also serve to fill in some easier proofs that are not provided in the main text.

\begin{exercise}\label{exer_bases}
	Prove from the definition of a base that if $\{ B_i \}_{i \in I}$ is a finite family of elements in some base $\mathcal{B}$ that have nonempty intersection then if $x \in B = \cap_{i \in I} B_i$ there exists some $B' \in \mathcal{B}$ such that $x \in B' \subset B$.
\end{exercise}





\chapter{Topological properties}
TODO: Stupid definitions?
\begin{define}{Topological property}{top_property}
	A predicate $P$ over topological spaces is called a \textbf{topological property} if homeomorphic spaces have the same truth value under $P$.
\end{define}
In simpler terms: $P$ is a topological property if whenever $X$ is homeomorphic to $Y$ we have that $X$ satistfies $P$ if and only if $Y$ satisfies $P$.

\begin{define}{Fully hereditary}{fully_hereditary}
	We say that a property $P$ is \textbf{fully hereditary} if whenever $X$ satisfies it then every subspace satisfies it too.
\end{define}

Rarely is a property fully hereditary.

\begin{define}{Weakly hereditary}{weakly_hereditary}
	We say that a property $P$ is \textbf{weakly hereditary} if whenever $X$ satisfies it then every closed subspace satisfies it too.
\end{define}

\begin{define}{Productory}{productory}
	We say that a property $P$ is \textbf{productory} if whenever a collection of spaces $\{ X_i\}_{i \in I}$ then the product $\Pi X_i$ satisfies it too.
\end{define}

\begin{define}{Strongly productory}{strongly_productory}
	We say that a property $P$ is \textbf{strongly productory} if it's productory and whenever a product of spaces satisfies $P$ then every factor in the product satisfies it too.
\end{define}

\section{Separation axioms}
\begin{define}{Separation axioms}{sep_axioms}
	Let $(X,\ \tau)$ be a topological space. We say that $X$ is
	\begin{itemize}
		\item
		$T_0$ (or \textbf{Kolmogorov}) if for all $x,y \in X$ distinct there exist an open set $U$ such that one of $x,y$ is in $U$ and the other one isn't.
		
		\item
		$T_1$ (or \textbf{Frechet}) if for all $x,y \in X$ distinct there exist 2 open sets $U$ and $V$ such that $x \in U, y \not \in U$ and $x \not \in V, y \in V$.
		
		\item
		$T_2$ (or \textbf{Hausdorff}) if for all $x,y \in X$ distinct there exist 2 open sets $U$ and $V$ such that $x \in U$, $y \in V$ and $U \cap V = \emptyset$.
		
		\item
		\textbf{regular} if for all $x \in X$ and closed sets $F \subset X$ with $x \not \in F$ there exist 2 open sets $U$ and $V$ such that $x \in U$, $F \subset V$ and $U \cap V = \emptyset$.
		
		\item
		$T_3$ if it's regular and $T_1$.
		
		\item
		\textbf{normal} if for all closed sets $F,G \subset X$ with $F \cap G = \emptyset$ there exist 2 open sets $U$ and $V$ such that $F \subset U$, $G \subset V$ and $U \cap V = \emptyset$.
		
		\item
		$T_4$ if it's normal and $T_1$.
	\end{itemize}
\end{define}

These are more-or-less standard notation. A bit less standard notation is these ones:

TODO: Does completely Hausdorff need to be T1? May be implied by the other part
\begin{define}{More separation axioms}{more_sep_axiomes}
	Let $(X,\ \tau)$ be a topological space. We say that $X$ is
	\begin{itemize}
		\item
		\textbf{completely Hausdorff} (or completely $T_2$) if it's $T_1$ and if for every $x,y\in X$ distinct there exists a continuous function $f: X \rightarrow \reales$ (in the euclidean topology) such that $f(x) = 0$ and $f(y) = 1$.
		
		\item
		\textbf{Tychonoff} (or \textbf{completely regular} or $T_{3\frac{1}{2}}$) if it's $T_1$ and if for every $x \in X$ and $C \subset X$ closed nonempty set with $x \not \in C$ there exists a continuous function $f: X \rightarrow \reales$ such that $f(x) = 1$ and $f(C) = \{ 0 \}$.
	\end{itemize}
\end{define}

\begin{theorem}
	$T_6 \Rightarrow T_5 \Rightarrow T_{4\frac{1}{2}} \Rightarrow T_4 \Rightarrow T_{3\frac{1}{2}} \Rightarrow T_3 \Rightarrow T_2 \Rightarrow T_1 \Rightarrow T_0$
\end{theorem}

It is of note that $T_3$ doesn't imply completely Hausdorff, as the next example implies:
\begin{example}

\end{example}

Tychonoff spaces are really interesting.

\begin{theorem}
	Let $X$ be a Tychonoff space with a base $\mathcal{B}$ of cardinal $\leq \kappa$. Then $X$ can be embedded in $[0,\ 1]^{\kappa}$.
\end{theorem}

\begin{theorem}
	Let $X$ be a topological space with weight $w(X) = \kappa$. Then $X$ is Tychonoff if and only if $X$ can be embedded in $[0,\ 1]^{\kappa}$.
\end{theorem}

\section{Cardinal invariants}
\begin{define}{Weight}{weight}
	Let $(X,\ \tau)$ be a topological space. We define the \textbf{weight} of $X$ to be the smallest infinite cardinal $\alpha$ for which there exists a base $\mathcal{B}$ of $X$ with $|\mathcal{B}| = \alpha$. It is denoted by $w(X)$.
\end{define}

\begin{define}{Character}{character}
	Let $(X,\ \tau)$ be a topological space. We define the \textbf{character of a point} $x\in X$ to be the smallest infinite cardinal $\alpha$ for which there exists a neighborhood base $\mathcal{B}$ of $x$ with $|\mathcal{B}| = \alpha$, and is denoted by $\chi(x, X)$. The \textbf{character} of $X$ is defined to be $\chi(X) = \sup\{\chi(x, X) : x \in X\}$.
\end{define}

Only allowing infinite cardinals makes arithmetic with these cardinal invariants a bit easier to handle.\\ 
There's a problem when working with these invariants: they require the axiom of choice to exist, as the next theorem entails:
\begin{theorem}
	(AC) Every set of cardinal numbers is well-ordered. That is to say it has a minimum element.
\end{theorem}
However, not every case needs the axiom of choice:

\begin{define}{Numerability axioms}{num_axioms}
\begin{itemize}
	\item
	If $w(X)$ is countable, then $X$ is said to be \textbf{2AN}.
	\item
	If $\chi(X)$ is countable, then $X$ is said to be \textbf{1AN}.
\end{itemize}
\end{define}

The countable cardinal is the minimum in any set of infinite cardinals, so for example we already know the set of the cardinals associated with the bases of a topological space that has a countable base will have countable weight.\\ 
As we will see when we talk about convergence, 1AN is an important axiom when talking about convergent sequences in an space.\\ 

\begin{theorem}
	$w(x) \geq \chi(X)$.
\end{theorem}

\begin{theorem}
	$\chi(X)|X| \geq w(x)$.
\end{theorem}

\begin{theorem}
	Let $\mathcal{B}$ be a base over $X$. Then there exists a base $\mathcal{B}' \subset \mathcal{B}$ such that $|\mathcal{B}'| = w(X)$.
\end{theorem}

\section{Connectedness}
\begin{define}{Connected space}{connected_space}
	Let $(X,\ \tau)$ be a topological space. We say that $X$ is \textbf{connected} if $X = A \cup B$ with $A,\ B \in \tau$ and $A \cap B = \emptyset$ imply that either $A = X$ or $B = X$. Equivalently, if the only partition of $X$ into two open disjoint sets is the trivial partition $\{ X,\ \emptyset\}$.
\end{define}

\begin{theorem}
	$X$ is connected if and only $\tau \cap \mathcal{F} = \{ \emptyset, X\}$ (the only open and closed sets are $\emptyset$ and $X$).
\end{theorem}

\begin{theorem}
	Being a connected space is a topological property.
\end{theorem}

The definition of connectedness concerns only partitions into 2 sets. We can expand it to arbitrary collections:

\begin{theorem}
	Let $(X,\ \tau)$ be connected and $X = \bigcup_{i \in I} U_i$ a partition of $X$ into disjoint open sets $U_i$, with $I$ an arbitrary index set. Then $U_i = \emptyset$ for all $i \in I$ but one $i_0 \in I$, for which we have $U_{i_0} = X$.
\end{theorem}

The converse is not needed.\\ 
When talking about the connectedness of subspaces associated with subsets of a topological space, we use the term connected set to mean that the subspace associated with it is connected.\\ 
Given connected subsets, we can obtain new connected sets from them:
\begin{theorem}
	Let $S$ be a connected set and $S \subset H \subset \closure(S)$. Then $H$ is connected.
\end{theorem}

\begin{theorem}
	Let $\{ S_i\}_{i \in I}$ be a collection of connected sets such that there exists $j \in I$ with $S_i \cap S_j \neq \emptyset$ for all $i \in I$. Then $\bigcup_{i \in I} S_i$ is connected.
\end{theorem}

\begin{define}{Connected component}{connected_component}
	Let $(X,\ \tau)$ be a topological space. A \textbf{connected component} of $X$ is a connected subspace $C \subseteq X$ such that for all $C \subsetneq A \subseteq X$ we have that $A$ is not connected.
\end{define}

\begin{theorem}
	Every connected component is a closed subset.
\end{theorem}

\begin{theorem}
	The set of all connected components of a space $X$ is a partition of $X$.
\end{theorem}

\begin{theorem}
	Let $x \in X$ be a point in a topological space. Then $x$ is contained in exactly one connected component, denoted by $C_x$ which is exactly the union of all connected subsets that contain $x$.
\end{theorem}

\begin{theorem}
	If $f: (X,\ \tau) \rightarrow (Y,\ \tau')$ is continuous then $f(C_x) \subset C_{f(x)}$.
\end{theorem}

\begin{theorem}
	The number of connected components is a topological property.
\end{theorem}

\begin{define}{Path connectedness}{path_connected}
	Let $(X,\ \tau)$ be a topological space. We say that $X$ is \textbf{path connected} if for all $x,\ y \in X$ there exists a continuous function $f: [0,\ 1] \rightarrow X$ such that $f(0) = x$ and $f(1) = y$
\end{define}

\begin{theorem}
	An open connected set in $\reales^n$ with the euclidean topology is path connected.
\end{theorem}



\section{Compactness}\label{sect_compactness}
\section{Miscellaneous properties}
Of course, we have not listed every topological property. Other, less important, examples of properties are listed here.
\begin{define}{Fixed point property}{fixed_point_property}
	An space $X$ is said to have the \textbf{fixed point property} (or to be \textbf{FPP}) if for every continuous function $f: X \rightarrow X$ there exists $x \in X$ such that $f(x) = x$ (that is to say every continous function has a fixed point). It is a topological property.
\end{define}

\begin{theorem}
	Every FPP space is connected.
\end{theorem}

The converse is false (as can been seen with $(0, 1)$ in the euclidean topology).\\ 
The problem of finding equivalent conditions for FPP is pretty hard.




\chapter{Metric spaces}
\section{Basic properties}
\section{Completeness}
\section{Metrizability}
\begin{define}{Metrizable space}{met_space}
	A topological space $(X,\ \tau)$ is said to be \textbf{metrizable} if there exists a metric $d$ on $X$ such that the topology generated by $d$ is equal to $\tau$. It is a topological property.
\end{define}




\chapter{Convergence and filters}

\chapter{Compactness}
Section \eqref{sect_compactness} gave an introduction to the notion of compactness. We provide a deeper exploration here.

\section{Stone-Čech compactification}
\begin{define}{C-embedded}{c_embedded}
	Let $X$ be a topological space, and $S \subset X$. We say that $S$ is \textbf{C-embedded} (resp. $C^*$-embedded) if every $f \in C(S)$ (resp. $C^*(S)$) extends to a function $g \in C(X)$ (resp. $C^*(X)$).
\end{define}

If $S$ is dense in $X$, then theorem (TODO: reference) tells us that the continuous extension is unique. This implies that $C(S)$ embeds into $C(X)$ as rings, via the application that sends each continuous function to it's unique extension.

\begin{theorem}[Urysohn extension]
	$S$ is a $C^*$-embedded subset of $X$ if and only if every pair of subsets that can be separated by continuous functions in $S$ can be separated by continuous functions in $X$.
\end{theorem}
\begin{proof}
	The direct implication is consequence of the definitions. For the converse, let $f$ be a continuous bounded function $f: S \rightarrow \reales$, let $\alpha > 0$ be an upper bound for the absolute value of $f$, and let $r_n = \frac{\alpha}{2} (\frac{2}{3})^n$ for any $n \in \naturales$. We notice that $|f(s)| \leq 3r_1$ for all $s \in S$. We search for an extension of $f$ to the entire space $X$. To that end, we proceed by induction. Suppose we have $n+1$ functions $f_m: S \rightarrow \reales$ for $m = 0$ integer to $n$ and another $n$ functions $g_m: X \rightarrow \reales$ for $m = 0$ integer to $n-1$ that satisfy
	\begin{equation*}
		|f_m(s)| \leq 3r_m\ \forall s \in S\ \forall m,
	\end{equation*}
	\begin{equation*}
		|g_m(x)| \leq r_m\ \forall x \in X\ \forall m,
	\end{equation*}
	\begin{equation*}
		f_{m+1} = f_m - g_m|_S\ \forall m \in \{ n-1, \ldots, 0\}.
	\end{equation*}
	Our base case is covered by letting $f_1 = f_0 = f$ and $g_0 = 0$. Let
	\begin{equation*}
		A_n = \{ x \in S : f_n(x) \leq -r_n\},
	\end{equation*}
	\begin{equation*}
		B_n = \{ x \in S : f_n(x) \geq r_n\}.
	\end{equation*}
	It is easily checked that $A_n$ and $B_n$ are separated by continuous functions in $S$, via the function (TODO: Write the function). By the hypothesis $A_n$ and $B_n$ are separated by continuous functions in $X$, and so there exists some continuous bounded function $g_n: X \rightarrow \reales$ such that $g_n(A_n) = \{ -r_n \}$, $g(B_n) = \{ r_n \}$ and $|g_n(x)| \leq r_n$ for all $x \in X$. We can now define the continuous function $f_{n+1}: S \rightarrow \reales$ via
	\begin{equation}\label{urysohn_ext_fdef}
		f_{n+1} = f_n - g_n|_{S},
	\end{equation}
	and notice that, for $x \in S$:
	\begin{enumerate}
		\item
		If $x \in A_n$ then $|f_{n+1}(x)| = |f_n(x) - g_n(x)| = |f_n(x) + r_n|\leq 2r_n$ because $-3r_n \leq f_n(x) \leq -r_n$.		
		
		\item
		If $x \in B_n$ then $|f_{n+1}(x)| = |f_n(x) - g_n(x)| = |f_n(x) - r_n| \leq 2r_n$ because $r_n \leq f_n(x) \leq 3r_n$.	
		
		\item
		If $x \not \in A_n$ and $x \not \in B_n$ then $|f_{n+1}(x)| = |f_n(x) - g_n(x)| \leq 2r_n$ because $f_n(x) \in ]-r_n, r_n[$ and $g(x) \in [-r_n, r_n]$.	
	\end{enumerate}
	So $|f_{n+1}(x)| \leq 2r_n = 2 * \frac{\alpha}{2} (\frac{2}{3})^n = 3r_{n+1}$ for all $x \in S$. By induction we have built two sequences of functions $f_n : S \rightarrow \reales$ with $|f_n(x)| \leq 3r_n$ and $g_n : X \rightarrow \reales$ with $|g_n(x)| \leq r_n$. By Weierstrass M-Test, the series $\sum_{n=1}^{\infty} g_n$ converges to some continuous function $g$ on $X$, and using \eqref{urysohn_ext_fdef} with some $x \in S$ we observe that:
	\begin{equation*}
		\sum_{n=1}^{m} g_n(x) = \sum_{n=1}^{m} (f_n(x) - f_{n+1}(x)) = f_1(x) - f_{m+1}(x).
	\end{equation*}
	Because of $|f_n(x)| \leq 3r_n$ we obtain that on $S$, $\sum_{n=1}^{m} g_n(x)$ converges to $f_1 = f$ and so $g = f$ on $S$.
\end{proof}


This theorem tells us that to prove that a subset is $C^*$-embedded we don't have to consider every mapping, we only have to consider the ones that separate sets.

\section{Rings of continuous functions}
\begin{define}{Rings of continuous functions}{cont_func}
	Let $X$ be a topological space, and 
	\begin{equation*}
		C(X) := \{ f: X \rightarrow \reales\ :\ f \text{ is continuous} \},
	\end{equation*}
	\begin{equation*}
		C^*(X) := \{ f: X \rightarrow \reales\ :\ f \text{ is continuous and bounded} \}.
	\end{equation*}
	Then $C(X)$ and $C^*{X}$ are rings with the usual addition and multiplication of functions. Moreover, $C^*(X)$ is a Banach algebra with the usual scalar multiplication and the norm
	\begin{equation*}
		\| f \| = \sup_{x \in X} | f(x) |,
	\end{equation*}
	white $C(X)$ is a complete metric space under the metric
	\begin{equation*}
		d(f, g) = \sup_{x \in X} \frac{| f(x) - g(x) |}{1 + |f(x) - g(x)|}.
	\end{equation*}
\end{define}

\chapter{Homotopy}


\begin{itemize}

\item
\textbf{Nets: }A net is a function from a directed set to a topological space. They are generalizations of sequences, and they are usually denoted by $\{x_{a}\}_{a\in A}$ where $A$ is a directed set.

\item
\textbf{General convergence: }A net $\{x_{a}\}_{a\in A}$  in a topological space $X$ converges to a point $x$ if for every neighborhood $U$ of $x$ there exists an element $b\in A$ such that for all $a\geq b$ we have $x_{a}\in U$. Note that in general spaces a net may converge to multiple points.

\item
\textbf{Limit point: }A point $x$ is called a limit point of a subset $A\subseteq X$ if every neighborhood of $x$ intersects $A$ at a point different from $x$. This is equivalent to there existing a net fully inside $A$ that converges to $x$.

\item
$\boldsymbol{T_{1}}$\textbf{ spaces: }A topological space $X$ is called a $\boldsymbol{T_{1}}$\textbf{ space} if every subset of $X$ that contains a finite amount of points is closed.

\item
\textbf{Hausdorff spaces: }A topological space $X$ is called a \textbf{Hausdorff space} if every 2 different points $x$ and $y$ can be separated by neighborhoods. That is, if there exists neighborhoods of $x$ and $y$ respectively that are disjoint whenever $x\neq y$.

\begin{theorem}[Unique convergence]
If a net $x_{a}$ converges in a Hausdorff space, then it converges to an unique point.
\end{theorem}
This explains to us why Hausdorff spaces are interesting to study.


\begin{theorem}[Closure and base]
Let $A\subseteq X$ and the topology on $X$ be given by a base. Then $x\in\bar{A}$ iff every basic neighborhood of $x$ intersects $A$.
\end{theorem}

\item
\textbf{Subbase of a topology: }A subbase $\delta$ for a topology over $X$ is a set of subsets such that their union is equal to $X$. We then get the topology generated by $\delta$ by grabbing all the unions of finite intersections of sets in $\delta$.

\item
\textbf{Subspace topology: }Let $X$ be a topological space and let $Y\subseteq X$. Then the set $T_{Y}=\{Y\cap U\ \forall U\in T \}$ is a topology over $Y$ called the \textbf{subspace topology}. We say that $Y$ \textbf{inherits} the topology from $X$.

\begin{theorem}[Open sets in subspaces relative to the super space]
Let $Y$ be an open subset in $X$. If $U$ is an open subset of $Y$ with the inherited topology from $X$ then $U$ is open in $X$ too.
\end{theorem}
If we change "open" to "closed" in that theorem it would still work.

\begin{theorem}[Closed sets in subspaces]
Let $Y$ be a subspace in $X$. Then a subset $A\subseteq Y$ is closed iff it is the intersection of a closed set in $X$ with $Y$.
\end{theorem}

\begin{theorem}[Closure in subspaces]
Let $Y$ be a subspace in $X$ and $A\subseteq Y$. Then $\bar{A}$ in $Y$ is equal to $\bar{A}\cap Y$ where this closure is in $X$.
\end{theorem}
We have to be careful whenever we are talking about an space and a subspace of it: a set may be closed or open in the subspace but not be in $X$, or viceversa.

\begin{theorem}[Subspaces of Hausdorff spaces]
If $X$ is a Hausdorff space, then any subspace of it is a Hausdorff space.
\end{theorem}

\item
\textbf{Ordered topology: }Let $(X, <)$ be an ordered set, and let $B$ be the collection of all open intervals of $X$ (the ones of the form $(a, b)= \{x|\ a<x<b\}$). If $X$ has a minimum add to $B$ intervals of the form $[a_{0}, b)$ for $a_{0}$ the minimum and $b$ any other element of the set and finally if $X$ has maximum, those intervals of the form $(a, b_{0}]$ where $b_{0}$ is the maximum. This forms a base for the \textbf{ordered topology} over a set. We can get a subbase for this topology by grabbing all \textbf{rays} in the set (intervals of the form $(-\infty, a)$ and $(a, +\infty)$ for a in the set).

\begin{theorem}[Subspaces of ordered spaces]
Let $X$ be an ordered space and $A$ be a convex subset of $X$ (that means that for every $a,b\in Y$ we have that $(a,b)\in Y$). Then, the subspace topology inherited from $X$ to $Y$ is the same as the ordered topology applied to $Y$.
\end{theorem}

\item
\textbf{Box and product topology: }Let $(X_{i})_{i\in I}$ be an arbitrary family of topological spaces. We can put a topology over ${\displaystyle \prod_{i\in I}X_{i}}$ two inequivalent ways:
\begin{enumerate}
\item
Take as a base every set ${\displaystyle \prod_{i\in I}U_{i}}$ where $U_{i}$ is open in $X_{i}$ for each $i\in I$. This topology is called the \textbf{box topology}.
\item
Take as a base every set ${\displaystyle \prod_{i\in I}U_{i}}$ where $U_{i}$ is open in $X_{i}$ for a finite amount of $i\in I$ and for every other $i$ let $U_{i}=X_{i}$. This topology is called the \textbf{product topology}.
\end{enumerate}
Both topologies are equal for finite products. For infinite products however, it's much more convenient to use the product topology instead of the box topology, and thus when we have a product of spaces we'll assume it has the product topology. In the infinite setting, both topologies are comparable and the box topology has more open subsets than the product topology.\\
If the topology on the original sets is given as a base, we can get bases for our product and box topology as follows:
\begin{enumerate}
\item
Take as a base every set ${\displaystyle \prod_{i\in I}B_{i}}$ where $B_{i}$ is basic in $X_{i}$ for each $i\in I$. This is a base for the box topology.
\item
Take as a base every set ${\displaystyle \prod_{i\in I}B_{i}}$ where $B_{i}$ is basic in $X_{i}$ for a finite amount of $i\in I$ and for every other $i$ let $B_{i}=X_{i}$. This is a base for the product topology.
\end{enumerate}

\begin{theorem}[Subspaces of product and box spaces]
If $A_{i}$ is a subspace of $X_{i}$ for every $i\in I$ then ${\displaystyle \prod_{i\in I}A_{i} }$ is a subspace of ${\displaystyle \prod_{i\in I}X_{i} }$ if both are given the box topology or the product topology.
\end{theorem}

\begin{theorem}[Product of Hausdorff spaces]
If $X_{i}$ is Hausdorff for every $i$ then ${\displaystyle \prod_{i\in I}X_{i} }$ is Hausdorff in both the box and product topologies.
\end{theorem}

\item
\textbf{Metric spaces: }Given a set $X$ and a metric $d$ on $X$, we can define a topology called the \textbf{metric topology} by taking as basic elements all the open balls of points of $X$ with radius bigger than 0. Then, a metric space is a set with this topology equipped with the metric that gives that topology. An space is called \textbf{metrizable} if there exists some metric on it's underlying set that gives the same topology.

\begin{theorem}[Bounded distance]
Let $X$ be a metric space equipped with the metric $d$. We define the metric $\bar{d}(x,y) = min (d(x,y), 1)$, which we call bounded metric for $d$. Then, $\bar{d}$ induces the same topology as $d$.
\end{theorem}

\begin{theorem}[Comparison of metric topologies]
Let $d$ and $d'$ be 2 metrics over a set $X$, and call the topologies that they generate $T$ and $T'$ respectively. Then, $T\subseteq T'$ iff for all $x\in X$ and $\varepsilon > 0$ there exists some $\delta > 0$ such that $B_{d'}(x,\delta)\subseteq B_{d}(x, \varepsilon)$
\end{theorem}

\begin{theorem}[Hausdorff condition in metric spaces]
Every metric space is Hausdorff.
\end{theorem}

\begin{theorem}[Subspaces of metric spaces]
Let $Y$ be a subset of a metric space $X$. Then $Y$ given the metric topology from the distance in $X$ is a subspace of $X$.
\end{theorem}

\begin{theorem}[Metrizability of products]
Let $(X_{i})_{i\in I}$ be a family of metric spaces over a countable index set. Then the product ${\displaystyle \prod_{i\in I}X_{i} }$ is metrizable. Moreover, there exists a metric that induces the same topology as the product topology.
\end{theorem}

\item
\textbf{Continuous functions: }A function between topological spaces $f:X\rightarrow Y$ is called continuous if any of the following equivalent conditions hold:
\begin{enumerate}
\item
For every open set $V$ in $Y$ the preimage of $V$ ($f^{-1}(V)$) is open in $X$.
\item
For every closed set $V$ in $Y$ the preimage of $V$ is closed in $X$.
\item
If the topology on $Y$ is given by a base, the preimage of any basic element is an open set in $X$.
\item
For every subset $A$ of $X$, we have $f(\bar{A})\subseteq \overline{f(A)}$
\end{enumerate}
We also have local conditions. Each of the above is equivalent to these holding for every point in the space, and a local condition holding for a point implies all the other conditions. For a point $x\in X$:
\begin{enumerate}
\item
For every neighborhood $V$ of $f(x)$ there exist a neighborhood $U$ of $x$ such that $f(U)\subseteq V$.
\item
For every net $x_{i}$ that converges to $x$ we have that the net $f(x_{i})$ converges to $f(x)$.
\end{enumerate}

\begin{theorem}[Operations on continuous functions]
Let $X,Y,Z$ be topological spaces:
\begin{enumerate}
\item
Let $f:X\rightarrow Y$ and $g:Y\rightarrow Z$ be continuous functions. Then $g\circ f$ is continuous.
\item
Let $f:X\rightarrow Y$ be continuous and let $A$ be a subspace of $X$. Then the restriction of $f$ to $A$ is continuous and denoted by $f|_{A}$.
\item
Let $f:X\rightarrow Y$ and let $B$ be a subspace of $Y$ containing $f(X)$. Then the restriction of $f$ to the range $B$ is continuous.
\item
Let $X=A\cup B$ where $A$ and $B$ are closed and $f:A\rightarrow Y$ and $g:B\rightarrow Y$ be two continuous functions that agree on $A\cap B$ (that means $\forall x \in A\cap B\ f(x)=g(x)$) then the function $h$ defined on $X$ by $h(x)=f(x)=g(x)\ \forall x \in A\cap B$, $h(y)=f(y)\ \forall y \in A-(A\cap B)$ and $h(z)=g(z)\ \forall z \in B-(A\cap B)$ is continuous.
\end{enumerate}
\end{theorem}

\begin{theorem}[Continuous functions on product spaces]
Let $f:A\rightarrow {\displaystyle \prod_{i\in I}X_{i} }$ be a function between topological spaces. We can see that $f(x)=(f_{i}(x))_{i\in I}$ for functions $f_{i}:A\rightarrow X_{i}$. Then, in the product topology $f$ is continuous iff $f_{i}$ is continuous for every $i$.
\end{theorem}

\item
\textbf{Homeomorphisms: }A continuous function is called a homeomorphism if it's bijective and it's inverse is also continuous. If $f:X\rightarrow Y$ is continuous and injective and the function obtained by restricting the range to $f(X)$ is a homeomorphism then $f$ is called an \textbf{embedding}.

\end{itemize}
\end{document}