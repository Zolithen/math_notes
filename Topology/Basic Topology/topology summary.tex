\documentclass[a4paper]{article}
\usepackage{amsmath}
\usepackage{amsfonts}
\usepackage{bm}
\usepackage[left=10pt,right=20pt,top=20pt,bottom=60pt]{geometry}

\newtheorem{theorem}{Theorem}[section]
\newtheorem{lemma}{lemma}[section]

\author{NyKi}
\title{Summary of Basic Topology}
\date{September 2024-January 2025}

\begin{document}
\maketitle
\tableofcontents
%%Thing
\section{Preface}
This file contains my notes for our \textit{Basic Topology} class at my university. I have decided to add more stuff I find interesting, but there's no distinction between what comes from class and what comes from my own research. Only proofs I find interesting are provided, and the notes are all over the place right now; I still need to sort them properly into chapters.
\section{Basic point set topology}
This is an introduction to topological concepts.
\begin{itemize}

\item
\textbf{Topological space: }A topological space is a pair $(X,T)$ where $X$ is a set and $T$ is a family of subsets of $X$ where:
\begin{enumerate}
\item
$X,\emptyset\ \in T$
\item
Any arbitrary union of elements of $T$ is again in $T$.
\item
Any finite intersection of elements of $T$ is again in $T$.
\end{enumerate}
Sets in T are called \textbf{open}. Sets which their complement is in $T$ are called \textbf{closed}. We usually call $X$ the topological space when we know the topology we are working with. Here, letters like $X$ and $Y$ will denote topological spaces. We also normally call an open subset that has a point $x$ a \textbf{neighborhood} of $x$.

\item
\textbf{Alternate definition of a topological space: }A topological space is a pair $(X, T)$ where $X$ is a set and $T$ is a family of subsets of $X$ where:
\begin{enumerate}
\item
$X,\emptyset\ \in T$
\item
Any finite union of elements of $T$ is again in $T$.
\item
Any arbitrary intersection of elements of $T$ is again in $T$.
\end{enumerate}
The sets in $T$ are called closed. Both of the definitions given are practically equivalent. People tend to use the open set definition of a topology, but these one gives us some "rules" to manipulate closed sets whenever we use the open set definition. There also exist multiple other definitions for a topology, like Kuratowski's closure operator definition, but those are equivalent to the first one in here and so are not considered.

\item
\textbf{Directed set: }A partially preordered set X is called directed whenever for every $a,b\in X$ we have that there exists a $c\in X$ such that $a\leq c$ and $b\leq c$.

\item
\textbf{Nets: }A net is a function from a directed set to a topological space. They are generalizations of sequences, and they are usually denoted by $\{x_{a}\}_{a\in A}$ where $A$ is a directed set.

\item
\textbf{General convergence: }A net $\{x_{a}\}_{a\in A}$  in a topological space $X$ converges to a point $x$ if for every neighborhood $U$ of $x$ there exists an element $b\in A$ such that for all $a\geq b$ we have $x_{a}\in U$. Note that in general spaces a net may converge to multiple points.

\item
\textbf{Limit point: }A point $x$ is called a limit point of a subset $A\subseteq X$ if every neighborhood of $x$ intersects $A$ at a point different from $x$. This is equivalent to there existing a net fully inside $A$ that converges to $x$.

\item
\textbf{Closure of a set: }Let $A\subseteq X$ be a subset. We define it's closure $\bar{A}$ as the smallest closed set that contains $A$. Equivalently, it's the intersection of all closed subsets of $X$ that contain $A$.

\begin{theorem}[Characterization of closure]
Let $A\subseteq X$ and $A'$ be the set of its limit points. Then $\bar{A}=A\cup A'$.
\end{theorem}
From this theorem we can get the result that $x\in \bar{A}$ iff every neighborhood of $x$ intersects $A$.

\item
$\boldsymbol{T_{1}}$\textbf{ spaces: }A topological space $X$ is called a $\boldsymbol{T_{1}}$\textbf{ space} if every subset of $X$ that contains a finite amount of points is closed.

\item
\textbf{Hausdorff spaces: }A topological space $X$ is called a \textbf{Hausdorff space} if every 2 different points $x$ and $y$ can be separated by neighborhoods. That is, if there exists neighborhoods of $x$ and $y$ respectively that are disjoint whenever $x\neq y$.

\begin{theorem}[Unique convergence]
If a net $x_{a}$ converges in a Hausdorff space, then it converges to an unique point.
\end{theorem}
This explains to us why Hausdorff spaces are interesting to study.

\item
\textbf{Base of a topology: }A base for a topology over a set $X$ is family of subsets B where:
\begin{enumerate}
\item
The union of all sets in B is equal to X.
\item
For every $x \in X$, if $x \in B_{1} \cap B_{2}$ for some $B_{1},B_{2} \in B$ then there exists a $B_{3} \in B$ such that $x \in B_{3} \subset B_{1} \cap B_{2}$
\end{enumerate}
A subset from the base is called a \textbf{basic element}. A basic element that contains a point $x$ is called a basic neighborhood.\\
From here, we can generate the topology T 2 equivalent ways:
\begin{enumerate}

\item
A subset $U$ of $X$ is called open iff for every $x \in U$ there exists a $B_{0} \in B$ such that $x \in B_{0} \subset U$.
\item
The topology T is the collection of all unions of subsets in B.
\end{enumerate}
\begin{theorem}[Going from topology to base]
Let $X$ be a topological space, and $C$ a collection of open subsets from $X$. If we have that for every open subset $U$ of $X$ and $x \in U$ there exists an element $C_{0} \in C$ such that $x \in C_{0} \subset U$ then $C$ is a base for the topology $T$.
\end{theorem}

\begin{theorem}[Comparison of topologies from bases]
Let $B_{1},B_{2}$ be bases for the topologies $T_{1},T_{2}$ respectively. The following are equivalent:
\begin{enumerate}
\item
$T_{1} \subset T_{2}$
\item
For every $x$ and subsets $b_{1} \in B_{1}$ that have $x \in b_{1}$ there exists a $b_{2} \in B_{2}$ such that $x \in b_{2} \subset b_{1}$.
\end{enumerate}
\end{theorem}

\begin{theorem}[Closure and base]
Let $A\subseteq X$ and the topology on $X$ be given by a base. Then $x\in\bar{A}$ iff every basic neighborhood of $x$ intersects $A$.
\end{theorem}

\item
\textbf{Subbase of a topology: }A subbase $\delta$ for a topology over $X$ is a set of subsets such that their union is equal to $X$. We then get the topology generated by $\delta$ by grabbing all the unions of finite intersections of sets in $\delta$.

\item
\textbf{Subspace topology: }Let $X$ be a topological space and let $Y\subseteq X$. Then the set $T_{Y}=\{Y\cap U\ \forall U\in T \}$ is a topology over $Y$ called the \textbf{subspace topology}. We say that $Y$ \textbf{inherits} the topology from $X$.

\begin{theorem}[Open sets in subspaces relative to the super space]
Let $Y$ be an open subset in $X$. If $U$ is an open subset of $Y$ with the inherited topology from $X$ then $U$ is open in $X$ too.
\end{theorem}
If we change "open" to "closed" in that theorem it would still work.

\begin{theorem}[Closed sets in subspaces]
Let $Y$ be a subspace in $X$. Then a subset $A\subseteq Y$ is closed iff it is the intersection of a closed set in $X$ with $Y$.
\end{theorem}

\begin{theorem}[Closure in subspaces]
Let $Y$ be a subspace in $X$ and $A\subseteq Y$. Then $\bar{A}$ in $Y$ is equal to $\bar{A}\cap Y$ where this closure is in $X$.
\end{theorem}
We have to be careful whenever we are talking about an space and a subspace of it: a set may be closed or open in the subspace but not be in $X$, or viceversa.

\begin{theorem}[Subspaces of Hausdorff spaces]
If $X$ is a Hausdorff space, then any subspace of it is a Hausdorff space.
\end{theorem}

\item
\textbf{Ordered topology: }Let $(X, <)$ be an ordered set, and let $B$ be the collection of all open intervals of $X$ (the ones of the form $(a, b)= \{x|\ a<x<b\}$). If $X$ has a minimum add to $B$ intervals of the form $[a_{0}, b)$ for $a_{0}$ the minimum and $b$ any other element of the set and finally if $X$ has maximum, those intervals of the form $(a, b_{0}]$ where $b_{0}$ is the maximum. This forms a base for the \textbf{ordered topology} over a set. We can get a subbase for this topology by grabbing all \textbf{rays} in the set (intervals of the form $(-\infty, a)$ and $(a, +\infty)$ for a in the set).

\begin{theorem}[Subspaces of ordered spaces]
Let $X$ be an ordered space and $A$ be a convex subset of $X$ (that means that for every $a,b\in Y$ we have that $(a,b)\in Y$). Then, the subspace topology inherited from $X$ to $Y$ is the same as the ordered topology applied to $Y$.
\end{theorem}

\item
\textbf{Box and product topology: }Let $(X_{i})_{i\in I}$ be an arbitrary family of topological spaces. We can put a topology over ${\displaystyle \prod_{i\in I}X_{i}}$ two inequivalent ways:
\begin{enumerate}
\item
Take as a base every set ${\displaystyle \prod_{i\in I}U_{i}}$ where $U_{i}$ is open in $X_{i}$ for each $i\in I$. This topology is called the \textbf{box topology}.
\item
Take as a base every set ${\displaystyle \prod_{i\in I}U_{i}}$ where $U_{i}$ is open in $X_{i}$ for a finite amount of $i\in I$ and for every other $i$ let $U_{i}=X_{i}$. This topology is called the \textbf{product topology}.
\end{enumerate}
Both topologies are equal for finite products. For infinite products however, it's much more convenient to use the product topology instead of the box topology, and thus when we have a product of spaces we'll assume it has the product topology. In the infinite setting, both topologies are comparable and the box topology has more open subsets than the product topology.\\
If the topology on the original sets is given as a base, we can get bases for our product and box topology as follows:
\begin{enumerate}
\item
Take as a base every set ${\displaystyle \prod_{i\in I}B_{i}}$ where $B_{i}$ is basic in $X_{i}$ for each $i\in I$. This is a base for the box topology.
\item
Take as a base every set ${\displaystyle \prod_{i\in I}B_{i}}$ where $B_{i}$ is basic in $X_{i}$ for a finite amount of $i\in I$ and for every other $i$ let $B_{i}=X_{i}$. This is a base for the product topology.
\end{enumerate}

\begin{theorem}[Subspaces of product and box spaces]
If $A_{i}$ is a subspace of $X_{i}$ for every $i\in I$ then ${\displaystyle \prod_{i\in I}A_{i} }$ is a subspace of ${\displaystyle \prod_{i\in I}X_{i} }$ if both are given the box topology or the product topology.
\end{theorem}

\begin{theorem}[Product of Hausdorff spaces]
If $X_{i}$ is Hausdorff for every $i$ then ${\displaystyle \prod_{i\in I}X_{i} }$ is Hausdorff in both the box and product topologies.
\end{theorem}

\item
\textbf{Metric spaces: }Given a set $X$ and a metric $d$ on $X$, we can define a topology called the \textbf{metric topology} by taking as basic elements all the open balls of points of $X$ with radius bigger than 0. Then, a metric space is a set with this topology equipped with the metric that gives that topology. An space is called \textbf{metrizable} if there exists some metric on it's underlying set that gives the same topology.

\begin{theorem}[Bounded distance]
Let $X$ be a metric space equipped with the metric $d$. We define the metric $\bar{d}(x,y) = min (d(x,y), 1)$, which we call bounded metric for $d$. Then, $\bar{d}$ induces the same topology as $d$.
\end{theorem}

\begin{theorem}[Comparison of metric topologies]
Let $d$ and $d'$ be 2 metrics over a set $X$, and call the topologies that they generate $T$ and $T'$ respectively. Then, $T\subseteq T'$ iff for all $x\in X$ and $\varepsilon > 0$ there exists some $\delta > 0$ such that $B_{d'}(x,\delta)\subseteq B_{d}(x, \varepsilon)$
\end{theorem}

\begin{theorem}[Hausdorff condition in metric spaces]
Every metric space is Hausdorff.
\end{theorem}

\begin{theorem}[Subspaces of metric spaces]
Let $Y$ be a subset of a metric space $X$. Then $Y$ given the metric topology from the distance in $X$ is a subspace of $X$.
\end{theorem}

\begin{theorem}[Metrizability of products]
Let $(X_{i})_{i\in I}$ be a family of metric spaces over a countable index set. Then the product ${\displaystyle \prod_{i\in I}X_{i} }$ is metrizable. Moreover, there exists a metric that induces the same topology as the product topology.
\end{theorem}

\item
\textbf{Continuous functions: }A function between topological spaces $f:X\rightarrow Y$ is called continuous if any of the following equivalent conditions hold:
\begin{enumerate}
\item
For every open set $V$ in $Y$ the preimage of $V$ ($f^{-1}(V)$) is open in $X$.
\item
For every closed set $V$ in $Y$ the preimage of $V$ is closed in $X$.
\item
If the topology on $Y$ is given by a base, the preimage of any basic element is an open set in $X$.
\item
For every subset $A$ of $X$, we have $f(\bar{A})\subseteq \overline{f(A)}$
\end{enumerate}
We also have local conditions. Each of the above is equivalent to these holding for every point in the space, and a local condition holding for a point implies all the other conditions. For a point $x\in X$:
\begin{enumerate}
\item
For every neighborhood $V$ of $f(x)$ there exist a neighborhood $U$ of $x$ such that $f(U)\subseteq V$.
\item
For every net $x_{i}$ that converges to $x$ we have that the net $f(x_{i})$ converges to $f(x)$.
\end{enumerate}

\begin{theorem}[Operations on continuous functions]
Let $X,Y,Z$ be topological spaces:
\begin{enumerate}
\item
Let $f:X\rightarrow Y$ and $g:Y\rightarrow Z$ be continuous functions. Then $g\circ f$ is continuous.
\item
Let $f:X\rightarrow Y$ be continuous and let $A$ be a subspace of $X$. Then the restriction of $f$ to $A$ is continuous and denoted by $f|_{A}$.
\item
Let $f:X\rightarrow Y$ and let $B$ be a subspace of $Y$ containing $f(X)$. Then the restriction of $f$ to the range $B$ is continuous.
\item
Let $X=A\cup B$ where $A$ and $B$ are closed and $f:A\rightarrow Y$ and $g:B\rightarrow Y$ be two continuous functions that agree on $A\cap B$ (that means $\forall x \in A\cap B\ f(x)=g(x)$) then the function $h$ defined on $X$ by $h(x)=f(x)=g(x)\ \forall x \in A\cap B$, $h(y)=f(y)\ \forall y \in A-(A\cap B)$ and $h(z)=g(z)\ \forall z \in B-(A\cap B)$ is continuous.
\end{enumerate}
\end{theorem}

\begin{theorem}[Continuous functions on product spaces]
Let $f:A\rightarrow {\displaystyle \prod_{i\in I}X_{i} }$ be a function between topological spaces. We can see that $f(x)=(f_{i}(x))_{i\in I}$ for functions $f_{i}:A\rightarrow X_{i}$. Then, in the product topology $f$ is continuous iff $f_{i}$ is continuous for every $i$.
\end{theorem}

\item
\textbf{Homeomorphisms: }A continuous function is called a homeomorphism if it's bijective and it's inverse is also continuous. If $f:X\rightarrow Y$ is continuous and injective and the function obtained by restricting the range to $f(X)$ is a homeomorphism then $f$ is called an \textbf{embedding}.

\end{itemize}

\end{document}