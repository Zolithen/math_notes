\documentclass{article}

\usepackage[spanish]{babel}
\usepackage{fontenc}
\usepackage[utf8]{inputenc}
\usepackage{hyperref}
\usepackage{graphicx}
\usepackage{amsmath}
\usepackage{amsfonts}
\usepackage{amssymb}
\usepackage{amsthm}
\usepackage[left=10pt,right=20pt,top=20pt,bottom=60pt]{geometry}

\newtheorem{theorem}{Teorema}
\newtheorem{prop}{Proposición}
\newtheorem{cor}{Corolario}
\newtheorem{axiom}{Axioma}
\newtheorem{define}{Definición}
\newtheorem{note}{Nota}

\author{NyKi}
\date{Diciembre 2024}

\begin{document}

\section{Introducción a los conjuntos numéricos}

\subsection{Construcciones}

\begin{note}
Las definiciones y construcciones de los conjuntos numéricos estándares aquí no se dan de una forma muy rigurosa. Su construcción es más propia de una asignatura de fundamentos matemáticos, y ahora mismo me da mucha pereza escribir todo.
\end{note}

Sea $\mathbb{N}$ un conjunto con un elemento que denominamos 1. Ahora, para todo elemento $n$ de $\mathbb{N}$ añadimos a $\mathbb{N}$ el sucesor, $S(n)$ o $n+1$. Esto da un conjunto infinito, los \textbf{números naturales}. En este conjunto tenemos el principio de inducción:

\begin{axiom}[Principio de inducción en $\mathbb{N}$]
Sea $S \subseteq \mathbb{N}$. Si $S$ satisface las siguientes 2 condiciones, entonces $S = \mathbb{N}$:
\begin{itemize}
\item
$1 \in S$
\item
$\forall n \in S \ n+1 \in S$
\end{itemize}
\end{axiom}

Este principio es muy útil para probar cosas sobre $\mathbb{N}$, por ejemplo la forma cerrada de una sucesión. En $\mathbb{N}$ también podemos definir algo denominado \textbf{orden total}, que es una relación binaria  $\leq$ que sigue los siguientes axiomas:
\begin{axiom}[Axiomas de orden total]
$\forall a,b,c \in \mathbb{N}$
\begin{enumerate}
	\item
	$a \leq a$ (Reflexividad)
	\item
	$a \leq b$ y $b \leq c$ implica $a \leq c$ (Transitivdad)
	\item
	$a \leq b$ y $b \leq a$ implica $a = b$ (Antisimetría)
	\item
	$a \leq b$ o $b \leq a$ (Totalidad)
\end{enumerate}
\end{axiom}

Cuando tenemos un orden parcial o total definido sobre un conjunto, podemos hablar de cotas y máximos y mínimos:
\begin{define}
Sea $S \subseteq X$ donde $X$ es un conjunto con un orden parcial o total $\leq$. $S$ es...
\begin{itemize}
\item
\textbf{Acotado superiormente} si $\exists r \in X$ tal que $x \leq r\ \forall x \in S$.
\item
\textbf{Acotado inferiormente} si $\exists r \in X$ tal que $r \leq x\ \forall x \in S$.
\end{itemize}
Y decimos que un elemento $r \in S$ es...
\begin{itemize}
\item
Un \textbf{máximo} si $\forall x \in S\ x\leq r$.
\item
Un \textbf{mínimo} si $\forall x \in S\ r\leq x$.
\end{itemize}
\end{define}


Con este orden total definido, podemos reformular el principio de inducción como:

\begin{axiom}[Principio de buena ordenación en $\mathbb{N}$]
\label{ax_wellordering}
$\forall S \subseteq \mathbb{N}\ S\neq \emptyset,\exists n \in S\ |\ \forall x \in S, n\leq x$. Es decir, todo subconjunto de los números naturales tiene mínimo.
\end{axiom}

Estas dos formulaciones son equivalentes.
Los números naturales además cumplen los siguientes axiomas algebraicos:

\begin{axiom}[Axiomas de semianillo unitario ordenado]
\label{ax_usemiring}
$\forall a,b,c \in \mathbb{N}$:
\begin{enumerate}
	\item
	$(a+b)+c=a+(b+c)$ (Asociatividad de la suma)
	\item
	$a+b=b+a$ (Conmutatividad de la suma)
	\item
	$(a*b)*c=a*(b*c)$ (Asociatividad de la multiplicación)
	\item
	$a*b=b*a$ (Conmutatividad de la multiplicación)
	\item
	$a*(b+c) = a*b + a*c$ (Distributividad de la multiplicación sobre la suma)
	\item
	$\exists\ 1 \in \mathbb{N}\ |\ \forall n \in \mathbb{N},1*n = n$ (Elemento neutro del producto)
	\item
	$a \leq b \ \Rightarrow \ a+c\leq b+c$ (Compatibilidad del orden con la suma)
	\item
	Si $c\geq 0 $ (que es trivial en $\mathbb{N}$), entonces $a \leq b\ \Rightarrow \ ac \leq bc $ (Compatibilidad del orden con el producto)
\end{enumerate}
\end{axiom}

Estos axiomas son particularmente débiles. Por ejemplo, para la ecuación $x + 2 = 4$ obviamente $x = 2$, pero no existe ninguna forma de probarlo fácilmente, cuando la existencia de inversos para cada número ayudaría inmensamente. Además, ecuaciones como $x + 4 = 2$ no tienen solución en $\mathbb{N}$. Por eso definimos un nuevo conjunto denominado $\mathbb{Z}$, los \textbf{números enteros}:

\begin{define}[Números enteros]
$\mathbb{Z} = \mathbb{N} \cup \{0\} \cup \{-n\ \forall n \in \mathbb{N} \}$ donde $0$ denota la identidad para la suma y $-n$ el inverso para la suma de $n$. 
\end{define}

Estos números, ademas de los \textit{Axiomas \ref{ax_usemiring}}, cumplen los siguientes axiomas:

\begin{axiom}[Axiomas adicionales para $\mathbb{Z}$]\ 
\label{ax_additionalz}
\begin{enumerate}
	\item
	$\exists\ 0 \in \mathbb{Z}\ |\ \forall n \in \mathbb{N},0+n = n$ (Elemento neutro de la suma)
	\item
	$\forall n \in \mathbb{Z},\exists -n \in \mathbb{z}\ |\ n+(-n)=0$ (Existencia del elemento inverso para la suma)
\end{enumerate}
\end{axiom}
Con estos axiomas, se dice que $(\mathbb{Z}, +)$ es un grupo conmutativo y que $(\mathbb{Z}, +, *)$ es un anillo conmutativo. A cambio de estos axiomas algebraicos, perdemos el principio de inducción en los números enteros y la existencia de un elemento mínimo, pero mantenemos una versión del principio de buena ordenación:

\begin{axiom}[Principio de buena ordenación de subconjuntos minorados de $\mathbb{Z}$]\ 
$\forall S \subseteq \mathbb{Z}\ S \neq \emptyset$ si $\exists n \in \mathbb{Z}\ |\ \forall x \in S,n\leq x$ entonces $\exists m \in S\ |\ \forall x \in S,m\leq x$. Es decir, todo subconjunto no vacío con cota inferior tiene un elemento mínimo.
\end{axiom}

Este axioma para $\mathbb{Z}$ implica el \textit{Axioma \ref{ax_wellordering}} para los naturales.
El conjunto de los números enteros aún tiene unos cuantos problemas. Por ejemplo, es imposible resolver la ecuación $2x=1$ para $x\in \mathbb{Z}$. Por eso, podemos definir otro conjunto de números construidos sobre los números enteros, los \textbf{números racionales}, denotados por $\mathbb{Q}$:

\begin{define}[Números racionales]
$\mathbb{Q} = \{ p/{q},p\in \mathbb{Z},q\in \mathbb{N}\}$
\end{define}

Aparte de cumplir los \textit{Axiomas \ref{ax_usemiring} y \ref{ax_additionalz}}, $\mathbb{Q}$ cumple:

\begin{axiom}[Axioma algebraico adicional para $\mathbb{Q}$]\ %fuck latex
\begin{enumerate}
	\item
	$\forall q \in \mathbb{Q}\ q \neq 0, \exists\ 1/q \in \mathbb{Q}\ |\ q * (1/q) = 1$ (Existencia del inverso de elementos no nulos para el producto)
\end{enumerate}
\end{axiom}

Esto hace de $\mathbb{Q}$ un cuerpo conmutativo. $\mathbb{Q}$ no tiene ni principio de buena ordenación, ni de buena ordenación de subconjuntos minorados (por ejemplo, el conjunto $S=\{ 1/n\ \forall n \in \mathbb{N}\}\subseteq \mathbb{Q}$ esta acotado inferiormente pero no tiene mínimo). Esto nos quita una vía de demostrar, pero "quitamos" más agujeros que existían en los números enteros:

\begin{theorem}[Densidad de $\mathbb{Q}$]
$\forall a,b \in \mathbb{Q}\ a \neq b,\ \exists r \in \mathbb{Q}\ |\ a<r<b$. Es decir, entre dos números racionales distintos siempre vamos a poder encontrar otro número racional. De hecho, vamos a poder encontrar infinitos aplicando el teorema cuantas veces como queramos.
\end{theorem}
\begin{proof}
Dados $a<b \in \mathbb{Q}$: $a = (a+a)/2 < (a+b)/2 < (b+b)/2 = b$. $(a+b)/2$ es el número que buscamos. 
\end{proof}
De este teorema podemos deducir que no existe una función sucesora en $\mathbb{Q}$, y por tanto no tenemos alternativa a inducción. Pero este teorema no es suficiente para que $\mathbb{Q}$ sea el conjunto numérico perfecto para hacer análisis. Aún existen agujeros, como demuestra el siguiente ejemplo:

\begin{prop}\label{prop_2_irrational}
No existe ningún $a \in \mathbb{Q}$ tal que $a^{2} = 2$.
\end{prop}
\begin{proof}
	Supongamos que $\exists a \in \mathbb{Q}$ tal que $a^2 = 2$. Al ser un número racional, lo podemos escribir de la forma $\frac{p}{q}$ con $p\in \mathbb{Z},q\in \mathbb{N}$ y $\gcd(p,q) = 1$ (donde $\gcd$ denota el máximo común divisor). Por tanto, tenemos la expresión $\frac{p^2}{q^2} = 2$, de donde deducimos que $p^2 = 2q^2$ y debido a que $2$ es un número primo, que $2|p$ o más concretamente $p=2k$ para algún $k \in \mathbb{Z}$. Substituyendo otra vez obtenemos $4k^2 = 2q^2$ y deducimos $2k^2 = q^2$, que de forma similar nos deja ver que $q$ es también múltiplo de 2. Pero inicialmente hemos asumido que el máximo común divisor de $p$ y $q$ es $1<2$ y no mayor o igual a $2$, por lo cual hemos encontrado una contradicción y la proposición es cierta.
\end{proof}
Esto es problemático, ya que intuitivamente deberíamos de poder encontrar un valor que cumpla $a^2 = 2$. Para poder arreglar este problema necesitamos una definición primero:

\begin{define}[Supremo e ínfimo]
	Sea $A$ un subconjunto numérico acotado superiormente. Si existe la mínima cota superior (es decir, un número $\omega$ que sea cota superior del conjunto y tal que cualquier otra cota superior $\alpha$ sea $\omega \leq \alpha$) esta será única y la llamaremos \textbf{supremo}. Dualmente, a la máxima cota inferior en un subconjunto acotado inferiormente la llamaremos \textbf{ínfimo}. Se denotan $\sup A$ y $\inf A$.
\end{define}
La definición parece ajena al ejemplo de ''agujero'' que hemos dado en la \textit{Proposición \ref{prop_2_irrational}}, pero es la más general que engloba todos los casos que necesitamos. El subconjunto $A\subseteq \mathbb{Q}$ definido como $A = \{a\in \mathbb{Q}\ |\ a^2 \leq 2\}$ esta acotado superiormente por $2$ y es posible demostrar que si existiera un supremo, este número sería tal que su cuadrado fuera igual a $2$, pero en $\mathbb{Q}$ no existe. Por tanto, podemos pensar que ''añadiendo todos los supremos'' completaríamos $\mathbb{Q}$. Este es el procedimiento que seguimos:
\begin{axiom}[Axioma del supremo] \label{ax_supr}
	Todo subconjunto acotado superiormente tiene supremo.
\end{axiom}
\begin{define}[Números reales]
	Al conjunto $\mathbb{R}$ con $\mathbb{Q} \subseteq \mathbb{R}$ y que cumpla el \textit{Axioma \ref{ax_supr}} lo llamamos los \textbf{números reales}.
\end{define}
Este conjunto no es único, pero si es único bajo isomorfismos, que viene a decir que cualesquiera dos conjuntos con estas propiedades tienen la misma estructura y por tanto no hace falta distinguirlos.

\subsection{Los números reales}
Empezamos el estudio de los números reales introduciendo algunos conceptos.
\begin{define}
El conjunto de los \textbf{irracionales} es $\mathbb{I} = \mathbb{R} \smallsetminus \mathbb{Q}$. Es decir, los números reales que no se pueden expresar como cociente de dos números enteros.
\end{define}

\begin{define}[Valor absoluto]
Dado $x \in \mathbb{R}$ definimos el \textbf{valor absoluto} como:
\begin{equation}
|x| = \left\lbrace
\begin{array}{l}
x\ \text{si}\ x \geq 0 \\
\\
-x\ \text{si}\ x < 0
\end{array}
\right.
\end{equation}
\end{define}
El valor absoluto es de los conceptos mas fundamentales del análisis real. Geométricamente, si dibujamos el valor $x \in \mathbb{R}$ en la recta real, el valor absoluto da la longitud del segmento que va desde $x$ hasta $0$. Geométricamente, la distancia entre dos números reales en la recta real viene dada por $d(x, y) = |x-y|$.
\begin{prop}[Propiedades del valor absoluto]
$\forall a,b \in \mathbb{R}$ tenemos:
\begin{enumerate}
\item
$|-a| = |a|$
\item
$|ab| = |a||b|$
\item
La desigualdad triangular: $|a+b| \leq |a| + |b|$
\item
La desigualdad triangular inversa: $||a|-|b|| \leq |a-b|$
\item
Si $a\neq 0$ entonces $|\frac{1}{a}| = \frac{1}{|a|}$
\end{enumerate}
\end{prop}
Usaremos mucho la desigualdad triangular. Cuando estudiemos sucesiones, nos será muy útil tener herramientas para relacionar los números naturales con los números reales.


\begin{theorem}[Propiedad arquimediana]
$\forall x,y \in \mathbb{R}$ tales que $x>0$ $\exists n \in \mathbb{N}$ con $nx>y$.
\end{theorem}
Podemos entender el teorema así: si tenemos una longitud muy pequeña siempre vamos a juntar muchas de ellas para poder formar una longitud grande.
\begin{cor}
El conjunto de los números naturales no está acotado superiormente.
\end{cor}
\begin{cor}
Todo subconjunto no vacío de los números naturales que esté acotado superiormente tiene máximo y mínimo.
\end{cor}

\begin{theorem}[Existencia de la parte entera]
$\forall x \in \mathbb{R}$ existe un único $k \in \mathbb{Z}$ tal que $k\leq x\leq k+1$.
\end{theorem}
\begin{define}[Parte entera]
Dado $x \in \mathbb{R}$ definimos la \textbf{parte entera} como:
\begin{equation}
[x] = \sup \{k\in \mathbb{Z}\ |\ k\leq x\}
\end{equation}
Por el teorema anterior, este supremo siempre existe.
\end{define}


\begin{theorem}[Existencia de raíces]
Sea $a \in \mathbb{R}$ cualquiera.
\begin{enumerate}
\item
Para todo $n \in \mathbb{Z}$ impar existe un único $x\in \mathbb{R}$ tal que $x^n = a$.
\item
Si $a\geq 0$ entonces para todo $n \in \mathbb{Z}$ par distinto de $0$ existe un único $x\in \mathbb{R}$ con $x\geq 0$ tal que $x^n = a$.
\end{enumerate}
\end{theorem}







\newpage
\section{Sucesiones de números reales}
\begin{define}
	Una \textbf{sucesión} es una aplicación $a_n: \mathbb{N} \rightarrow \mathbb{R}$. El conjunto $\{a_n\}_n \subseteq \mathbb{R}$ es el conjunto imagen. Decimos que la sucesión
	$\{a_n\}_n \subseteq \mathbb{R}$ es:
	\begin{itemize}
		\item
		\textbf{Eventual creciente} si $\exists N \in \mathbb{N}$ tal que $a_m \geq a_n,\ \forall m>n\geq N$ con $n$ y $m$ naturales.
		\item
		\textbf{Eventual estrictamente creciente} si $\exists N \in \mathbb{N}$ tal que $a_m > a_n,\ \forall m>n\geq N$ con $n$ y $m$ naturales.
		\item
		\textbf{Eventual decreciente} si $\exists N \in \mathbb{N}$ tal que $a_m \leq a_n,\ \forall m>n\geq N$ con $n$ y $m$ naturales.
		\item
		\textbf{Eventual estrictamente decreciente} si $\exists N \in \mathbb{N}$ tal que $a_m < a_n,\ \forall m>n\geq N$ con $n$ y $m$ naturales.
	\end{itemize}
	Si $N = 1$ en cualquiera de estos casos, entonces quitamos ''eventual".
\end{define}
Nos interesa estudiar el comportamiento de la sucesión cuando $n$ se hace grande, es decir \textit{eventualmente}. Veremos que el comportamiento de $a_n$ cercano a $n = 1$ no importa
en lo que a límite se refiere, solo el comportamiento eventual.


\begin{define}
	Sea $\{ a_n\}_n \subseteq \mathbb{R}$ una sucesión de números reales. Decimos que $a_n$ \textbf{converge} a $L$ y lo denotamos como $a_n \rightarrow_{n} L$ si y solo si 
	$\forall \varepsilon > 0,\exists N \in \mathbb{N}$ tal que $|a_n - L| < \varepsilon$ para todo $n$ tal que $n \geq N$.
\end{define}


\begin{prop}[Unicidad del límite]
	Sean $\{ a_n\}_n \subseteq \mathbb{R}$ convergente a $L_1$ y $L_2$. Entonces $L_1 = L_2$.
\end{prop}
\begin{proof}
	Supongamos que $L_1$ y $L_2$ son dos límites de la sucesión $\{ a_n\}_n \subseteq \mathbb{R}$. Por tanto, dado un $\varepsilon > 0$, existe un $N \in \mathbb{N}$ 
	tal que $|a_n - L_1| < \varepsilon/2$ y $|a_n - L_2| < \varepsilon/2\ $ $\forall n > N$ con $N=max\{n_1, n_2\}$, donde $n_1$ y $n_2$ son $N$ dado de la definición del límite para 
	$L_1$ y $L_2$. Sumando ambas desigualdades y usando la desigualdad triangular, tenemos:
	\begin{align*}
		\varepsilon = \frac{\varepsilon}{2} + \frac{\varepsilon}{2} > |a_n - L_1| + |a_n - L_2| = |a_n - L_1| + |-a_n + L_2| \\ \geq |a_n - L_1 - a_n + L_2| = |L_2 - L_1|
	\end{align*}
	Como $\varepsilon$ es un número arbitrario mayor que 0, si asumimos que $|L_2 - L_1| \neq 0$, siempre vamos a poder encontrar un valor de $\varepsilon$ (por ejemplo, 
	$\varepsilon = \frac{|L_2 - L_1|}{2} > 0$) mayor a 0 que contradiga $\varepsilon > |L_2 - L_1|$. Por tanto, $|L_2 - L_1| = 0$ y $L_2 = L_1$.
\end{proof}


\begin{define}
	Sabiendo que una sucesión $\{ a_n\}_n \subseteq \mathbb{R}$ converge a un único valor $L$, llamamos a este valor \textbf{límite de la sucesión $a_n$}, y lo denotamos como
	$\lim_{n\rightarrow +\infty} a_n = L$ o de manera resumida $\lim_{n} a_n = L$.
\end{define}


\begin{define}
	Si $\{ a_n\}_n \subseteq \mathbb{R}$ no converge a ningún valor $L\in\mathbb{R}$, podemos decir que $a_n$ es:
	\begin{itemize}
		\item
		\textbf{Divergente a $+\infty$} si $\forall M \in \mathbb{R},\ \exists N \in \mathbb{N}$ tal que $a_n > M\ \forall n>N$.
		
		\item
		\textbf{Divergente a $-\infty$} si $\forall M \in \mathbb{R},\ \exists N \in \mathbb{N}$ tal que $a_n < M\ \forall n>N$.
		
		\item
		\textbf{Oscilante} si no converge ni diverge.
	\end{itemize}
\end{define}


\begin{define}
	Sea $\{ n_k\}_k \subseteq \mathbb{N}$ una sucesión estrictamente creciente de números naturales y $\{ a_n\}_n \subseteq \mathbb{R}$ una sucesión cualquiera de números reales.
	Una \textbf{subsucesión de $a_n$} es una sucesión de la forma $\{ a_{n_k}\}_k \subseteq \mathbb{R}$.
\end{define}

\begin{theorem}[Aritmética de límites]
	Sean $\{ a_n\}_n \subseteq \mathbb{R}$ y $\{ b_n\}_n \subseteq \mathbb{R}$ dos sucesiones cualesquiera convergentes a $L_a$ y $L_b$ respectivamente. Entonces:
	\begin{itemize}
		\item
		Si $r \in \mathbb{R}$, entonces $\lim_{n\rightarrow +\infty} ra_n = rL_a$
		\item
		$\lim_{n\rightarrow +\infty} a_n + b_n = L_a + L_b$
		\item
		$\lim_{n\rightarrow +\infty} a_nb_n = L_aL_b$
		\item
		Si $b_n \neq 0\ \forall n \in \mathbb{N}$, entonces $\lim_{n\rightarrow +\infty} \frac{a_n}{b_n} = \frac{L_a}{L_b}$
	\end{itemize}
\end{theorem}










\section{Funciones, límites y continuidad}

\section{Derivabilidad de funciones reales}

\end{document}
